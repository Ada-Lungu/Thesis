%!TEX root=thesis.tex

\chapter{The User Experience}

% Command - B = compile

\section{The Target Audience}

The end-users of our product are children with ages between 8 and 15 years old. Part of our development strategy was a continuous testing of the platform with representatives of these group age. We have set our testing environment through the workshops in which we have been assisting as tutors. 
We have desiged the platform iteratively based on continuous feedback from our focus groups. We have redesigned features in accordance with free observations of users interacting with our product and with direct feedback from our end-users. \\

The platform has distinct end-users, the main target, the kids, and the administrators or tutors of the workshops. Consequently, the application has two modes and we have adopted different presentation styles for the two distinct groups of users. Our focus was mainly on designing a friendly user interface for the main target which are the kids, and invested less effort in polishing the administrator panel. 
All in all, we have kept a minimalistic design for the whole app, both as an aesthetic principle and as a strategy for further future graphics integration.
In the next subsections of the chapter I will briefly describe what are the main features of each end-users mode.

\subsection{Kids}
formative assessment
\subsection{Admin Panel}

We have designed the administration panel system in such a way that admins can manage the platform content and monitor users activity. We cover the management of all components of the application from the admin panel. \\

There's the Projects Management section, where admins have an overview of all projects. Here, they have the choice to activate or desactivate a project, meaning the project will get displayed on the projects page or will be hidden. Admins can edit the presentation data of the projects or can delete it from the system. 
Project management panel has a users tracking progress feature, which gives admins an overview of projects and users progress level.
[Screenshot]\\

The Quiz Management system allows adding new quizes: uploading images and creating multiple choice answers, setting the correct answers etc. It has the possibility of enabling or disabling quizzes. \\

User Management: reset passwords

Equipment Management

Chat Control


\section{Typical Usecase for Kids/Persona Scenario}


\subsection{Account Creation}

\subsection{Learning by doing: Projects}

An important section of the platform is dedicated to Projects. Projects are a central component of the learning experience on the platform. \\

Projects Section is split in the following subsections. The main page, where all projects are diplayed and users can search by labeled categories. projects are briefly presented by a distinctive image and their title. Here, on the main page, users are also able to add a new project, which will bring them to the editing mode (Edit Page) of projects. \\

The editing or creation mode is currently open to administrators of the platform for putting up and editing teaching materials related to the project: title, description, media resources, a list of needed materials/items. This mode supports uploading of files (pdf), images (png, jpeg) and videos.
Kids can create projects as well, so that the users become active contributers to the platform. To ensure safety/appropriate content, the uploads and data of the kids await check-ups and approval from the administrators before getting uploaded on the website.
In the end we want the platform to be a place where the user is also author and contributor on the platform. \\

In the project presentation mode, the user will find all details about the current selected project: description, label tags, needed materials or items list (estimated time of completion). The materials list come along with an order and purchase system. Users can check the items they already have and can add to basket for purchase the ones they need. 
Users can leave comments on the page, with questions or any information related to the project. The page includes a stars voting system, where they can rate the project based on a few criterias: functionality, difficulty, fun. (Users will be permitted to vote only after completing the project.)\\

 Each project is split in a number of steps, that the user must complete, in order to get the project part of his portofolio.

 [Tracking progress feature]


\subsection{Feedback: Gamification, Rankings, Achievements}

The application is built around a motivational reward system. It uses gamification techniques to transform learning into an engaging and provocative activity. \\

The reward system is represented by an achievements and points system. 
Achievements have two states: locked and unlocked. They are split in categories, each category standing for certain types of activitis and tasks that the participants can perform. Initially all achievements are locked and they can be activated only when users perform specific actions.\\

The points system is linked to the rankings section which enhance a competitive experience for the users. 
There are 3 types of rankings: general - points per user, users are ranked based on individual achievements in a list with all active users; per workshop - rank individual achievements between members of same workshop; per groups - rank collective performance of workshops based on the average points gained by their members. \\

We have intended to design the rankings in a way that stimulates both an individual and collective competitiveness and performance. The rankings per groups targets to boost children spirit of collaboration and feeling of group adherance. 

In the current beta version, there are 2 functional categories of achievemnts: the the Quiz Achievemnts and the General Achievemnts. The achievements are closely linked to the points system, each coming with a number of points as a reward.\\

A central gamification feature intended for the release version will integrate and link together the current gamification elements by introducing an avatar buiding system. Users will build their avatars trading their achievements and points for items that will constitute parts or accesories of the avatar.

\subsection{Assessment: Quizzes}

The core feature on the current version, generator of points and achievemnets are the quizzes. Quizzes are linked to workshop themes and projects section and assess users level of mastering a topic. They have multiple choise answers with one correct answer. Users users can evaluate their answers and visualize their wrong answers against the correct ones. They are evaluated by the number of correct answers and by the time of quiz completion. They can redo the quiz to upgrade the completion time, but their ranking position is given by their first quiz completion. \\

In the future release version, quizzes will be locked by default and will be unlocked only in a logical sequence. 


\section{Usability Tests}




















