%!TEX root=thesis.tex

\chapter{User Experience}

% Command - B = compile

\section{Target End-Users Of the Application}

The main end-users of our product are children with ages between 8 and 15 years old. Part of our development strategy was a continuous testing of the platform with representatives of this group age. We have set our testing environment through the workshops in which we have been assisting as tutors. 
We have designed the platform iteratively based on continuous feedback from our target users. We have redesigned features in accordance with free observations of users interacting with our product and with direct feedback from them. \\

The other \ins{category of} end-users for the platform are administrators, which at the moment are represented by the tutors of the workshops. By playing both the role of a tutor and the developer of the application, I had the chance in directly testing the product usability \ins{with respect the administrator role }in its real life context.\\

The profile and the needs of two types of end-users appeal for quite different presentation styles. Consequently, the application has two modes for the two target groups. We focused in designing a user friendly interface for the main target which are the kids, and invested less effort in polishing the administrator panel.	\\

As a central user experience convention, we have kept a minimalistic design for the whole app, both as an aesthetic principle and as a strategy for \ugh{further future} graphics integration.	
In the next subsections of the chapter I will briefly describe what are the main features of each \ins{of the two }end-user modes.


\newpage
\section{Typical Usecase for Kids}

\ins{The platform supports a series of use cases for kids}
\ml{It's nice to have some text between sectin and subsection, chapter and section, etc.}


\subsection{Finding A Project}
\ml{the paranthesis shows you're undecided. choose one only!}

Projects Section is \ugh{a central component}\ml{the entry point maybe?} of the learning experience on the platform.
The main page of the module displays all projects to users. Projects are briefly presented by a distinctive image and their title. [Figure ~\ref{fig:projects_display}]\\

\begin{figure}[h!]
\includegraphics[width=1\linewidth]{images/ui/DisplayProjects.png}
\caption{Main Projects Page - Projects Overview.}
\label{fig:projects_display}
\end{figure}


\ins{To accomodate a future where the platform will host a large number of projects, we implemented also } a filtering feature for projects \ins{that allows the projects to be displayed based on} labeled categories. [Figure ~\ref{fig:labels}] 

\begin{figure}[h!]
\centering
\includegraphics[width=\linewidth]{images/ui/Labels.png}
\caption{Filtering Category Labels for Projects.}
\label{fig:labels}
\end{figure}


\subsection{Creating and Editing a Project}
Users can create and edit a project. The editing mode is currently open to administrators of the platform for \chg{putting up}{uploading} and editing teaching materials related to the project: title, description, media resources, a list of needed materials. This mode supports uploading of files (pdf), images (png, jpeg) and videos.

Kids can create projects as well, so that they become active contributors to the platform. To ensure appropriate content, the data submitted by kids await check-ups and approval from the administrators before getting uploaded on the website.


\subsection{Interacting With a Project}
In the project presentation mode, the user finds all details about the current selected project: description, label tags, needed materials, estimated time of completion. The materials list come along with an order and purchase system. Users can check the items they already have and can add to basket for purchase the ones they need. 

Each project is split in a number of steps that the user must complete, in order to get rewards such as points and achievements. The user is able to visualize the progress of steps completion on the tracking progress bar at the top of the steps view. [Figure ~\ref{fig:project_steps}] 

\begin{figure}[h!]
\includegraphics[width=\linewidth]{images/ui/ProjectSteps.png}
\caption{Project Steps Presentation.}
\label{fig:project_steps}
\end{figure}



Users can leave comments on the page, with questions or any information related to the project. [Figure ~\ref{fig:comments}] 

\begin{figure}[h!]
\includegraphics[width=1\linewidth]{images/ui/Comments.png}
\caption{Users Comments System.}
\label{fig:comments}
\end{figure}

The page includes a stars voting system, where they can rate the project based on a few criterias: functionality, difficulty, fun. As a future feature, users will be permitted to vote only after completing the project. [Figure ~\ref{fig:stars_voting_system}] 

\ml{daca imaginea nu este full-width, atunci potzi sa folosesti comanda centering ca sa o centrezi in pagina orizontal }
\begin{figure}[h!]
\centering
\includegraphics[width=0.5\linewidth]{images/ui/StarsVotingSystem.png}
\caption{Stars Voting System.}
\label{fig:stars_voting_system}
\end{figure}


Moreover, they can contribute to extending the labels decribing the project, by adding new descriptive tags. [Figure ~\ref{fig:add_labels}] \\

\begin{figure}
\centering
\includegraphics[width=0.5\linewidth]{images/ui/AddLabels.png}
\caption{Add Labels for Project.}
\label{fig:add_labels}
\end{figure}





\subsection{Learning through Play: Gamification, Rankings, Achievements}

\ml{ar fi fain daca titlul sectiunii ar intra pe un rand}
\ml{si in sectiunea asta ar fi fain daca te-ai juca putzin cu pozitionarea imaginilor in asa fel incat ele sa fie inserate in text la locul unde se vorbeste despre ele.}
\ml{Ai putea sa ai doua sectiuni de fapt: Gamification: Achievements si Gamification: Points. Maybe :)}

The application is built around a motivational reward system. It uses gamification techniques to transform learning into an engaging and provocative activity. \\


The reward system is represented by achievements and a points system. Achievements have two states: locked and unlocked. They are split in categories, each category standing for certain types of activities and tasks that the participants can perform. Initially all achievements are locked and they can be activated only when users perform specific actions. The locked state of achievements is visually represented by a grey overlay or background (depending on the context where they are displayed). The unlocking of an achievement is marked by a pop up in notifications section on the menu. [Figure ~\ref{fig:AchievementsPopUps}] The achievements have a general presentation page and can be found also on the user dashboard, where only a maximum of 6 achievements are displayed (always a few unlocked for challenging the user). [Figure ~\ref{fig:AchievementsDashboard}] \\

In the current beta version, there are 2 functional categories of achievements: the Quiz Achievements and the General Achievements. [Figure ~\ref{fig:GeneralAchievements}] The achievements are closely linked to the points system, each coming with a number of points as a reward.\\

\begin{figure}
\includegraphics[width=1\linewidth]{images/ui/GeneralAchievements.png}
\caption{General Achievements.}
\label{fig:GeneralAchievements}
\end{figure}

\begin{figure}
\includegraphics[width=1\linewidth]{images/ui/AchievementsDashboard.png}
\caption{Achievements Display on Dashboard.}
\label{fig:AchievementsDashboard}
\end{figure}

\begin{figure}
\includegraphics[width=1\linewidth]{images/ui/AchievementsPopUps.png}
\caption{Achievements Notifications.}
\label{fig:AchievementsPopUps}
\end{figure}


The points system is linked to the rankings section which enhance a competitive experience for the users. 
There are 3 types of rankings: \ml{cand spui three types, este un hint ca ai nevoie de o enumerare :) } 

\begin{enumerate}
	\item \textit{general rankings} - rank individual achievements comparing the performance of all active users
	\item \textit{workshop rankings} - list individual achievements between members of same workshop
	\item \textit{groups rankings} - evaluate collective performance per workshops based on the average points gained by their members. 
\end{enumerate}

We have designed the rankings in a way that stimulates both an individual and collective competitiveness. The rankings per groups targets to boost children spirit of collaboration and feeling of group adherance. \\

During our UX testing sessions with the kids, we have concluded that the rankings is the most popular feature of the platform, as kids kept coming back and checking the rankings page many times per session. Rankings motivates kids in taking more and more quizzes and redoing quizzes for gaining more points.
Based on our UX testing findings, we included a number of links to rankings on the platform: when logging in on dashboard page, after the completion of a quiz. [Figure ~\ref{fig:Rankings}] \ml{Figura nu ilustreaza sentence-ul. Si avetzi Links sau pur si simplu p}
\\

\begin{figure}
\includegraphics[width=1\linewidth]{images/ui/Rankings.png}
\caption{Ranking System.}
\label{fig:Rankings}
\end{figure}

A central gamification feature intended for the release version will integrate and link together the current gamification elements (achievements and points) by introducing an avatar buiding system. Users will build their avatars trading their achievements and points for items that will constitute parts or accessories of the avatar. We take into consideration adopting a monetary system represented by coins, with an equivalent of points per coin.

\ml{If this is future work, it could also be moved to the coresponding section ;)}


\subsection{Assessment Techniques: Quizzes}
\ml{this is still a subsection of Typical Use Cases for Kids. Thus the title should be something like "Taking a Quiz"}

The main feature of the current version that generates points and unlock achievements are quizzes. Quizzes are linked to workshop themes and to projects and assess users level of mastering a topic.
After completing a quiz, users can evaluate their answers and visualize the wrong answers against the correct ones. [Figure ~\ref{fig:Quizzes}]

Users are evaluated by the number of correct answers and by the time of quiz completion. They can redo the quiz to upgrade the completion time, but they can't update their points, which are decided by their first quiz completion. \\

In a future release version, quizzes will be locked by default and will be unlocked only in a logical sequence. 

\begin{figure}
\includegraphics[width=1\linewidth]{images/ui/Quizzes.png}
\caption{Quizzes Overview.}
\label{fig:Quizzes}
\end{figure}

\section{Typical Usecase For Admins}

We have designed the administration panel system in such a way that admins can manage most of the platform content and monitor users activity. From the admin panel, administrators can add and edit content displayed on the platform (quizzes, projects), activate or disable components (the chat system), view and manage users profiles and activity (login history, reset passwords). Each main section on the platform has a correspondent management control in the admin panel. \\



\begin{figure}
\includegraphics[width=1\linewidth]{images/ui/AdminPanelMenu.png}
\caption{Admin Panel Menu.}
\label{fig:AdminPanelMenu}
\end{figure} 

In the Projects Management section, the admins have an overview of all projects, can edit the display data of the projects or can delete projects from the system. Projects can be
disabled or turned on, meaning admins choose which projects are currently displayed on the projects page. 
Project Management panel includes an users progress tracking feature, which gives admins an overview of the projects started by users and the level of completion progress.
\\

\begin{figure}
\includegraphics[width=1\linewidth]{images/ui/AdminPanelProjects.png}
\caption{Admin Panel Projects Control.}
\label{fig:AdminPanelProjects}
\end{figure} 

The Quiz Management system allows adding or editing quizzes: uploading images and creating multiple choice answers, setting the correct answer. As it does in projects, the admin system allows the enabling or disabling of quizzes.\\

The profiles and activity of users can be monitored and managed in the User Management panel. The Login History comprises the timestamps of the users having logged in on the application. The Reset Password feature came as a result of a common problem tutors faced in the actual workshops: kids frequently forgetting their credentials. The Group Management is also linked to the real needs of the workshops. We want facilitate the tutors in changing names or adding new workshops, without the need of the developer changing entried in database. \\

In the Chat Control panel, tutors can disable the chat. We have designed this feature, after chatting during the workshops became a way of distracting kids from focusing on the projects and paying attention. 
\ml{This is a cool observation. It would be nice if you had a special tag for this kind of lessons learned. Sau ai putea sa aduni toate lectziile din nou la sfarshit. Mai era una faina, aia cum ca copii sunt cel mai motivatzi de puncte si de rankings.}






















